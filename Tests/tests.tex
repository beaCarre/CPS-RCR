\documentclass[a4paper, 11pt, notitlepage]{report}

\setlength{\hoffset}{-1.5cm}
\setlength{\voffset}{-1.5cm}
\setlength{\textwidth}{18cm}
\setlength{\textheight}{24cm}
\setlength{\oddsidemargin}{0pt} % Marge gauche sur pages impaires
\setlength{\evensidemargin}{0pt} % Marge gauche sur pages paires

\usepackage[utf8]{inputenc}
\usepackage[frenchb]{babel}
\usepackage[T1]{fontenc}
\usepackage{graphicx}
\usepackage{amsmath}%\overset{min}{\eq}
\usepackage{hyperref}
\usepackage{lscape}

%\usepackage{csquotes}

\usepackage{listings}
\usepackage{color}


\definecolor{lightgray}{rgb}{.9,.9,.9}
\definecolor{darkgray}{rgb}{.4,.4,.4}
\definecolor{purple}{rgb}{0.65, 0.12, 0.82}


  \lstdefinelanguage{testsgrammar}{
  morekeywords={Cas, de, test, CI, Operation, Oracle},
}
 \lstnewenvironment{Test}
		  {\lstset{
		      language=testsgrammar,
		      breaklines=true,
		      showstringspaces=false,
		      keywordstyle=\color{blue},
		      identifierstyle=\footnotesize,
		      basicstyle=\footnotesize,
		      escapeinside={(*}{*)},
		      tabsize=3,
		      %xleftmargin=0.01\textwidth
		    }
		  }
		  {} 
\newcommand{\specB}[1]{\textbf{#1}}
\title{
  \huge Tests \\
}
\author{
  }
\date{}

\begin{document}

\maketitle

%% \section*{Introduction}
%% Lien vers l'énoncé du projet : 
%% \href{http://www-master.ufr-info-p6.jussieu.fr/2013/spip.php?action=acceder_document&arg=2148&cle=11624d21d0734169986d403a88c3f6e4e1755b65&file=pdf\%2Fsujet_projet_2014.pdf}{lien}.

\section{Le service Personnage}

\begin{Test}
Cas de test : Personnage::testInitWorking
CI : nom = "Ryan" (*$\land$*) l = 30 (*$\land$*) h = 30 (*$\land$*) p = 30 (*$\land$*) a = 10 (*$\land$*) v = 100 (*$\land$*) f = 100
Operation : P0 (*$\stackrel{def}{=}$*) init(nom,l,h,p,f,v,a)
Oracle : 
	nom(P0) == "Ryan"
	largeur(P0) =	30
	profondeur(P0) = 30
	hauteur(P0) = 30
	pointsDeVie(P0) = 100
	force(P0) = 100
	argent(P0) = 10
	
	
Cas de test : Personnage::testInitFailing 
CI : nom = "Joe" (*$\land$*) l = 30 (*$\land$*) h = 30 (*$\land$*) p = 30 (*$\land$*) a = 10 (*$\land$*) v = 100 (*$\land$*) f = 100
Operation : P0 (*$\stackrel{def}{=}$*) init(nom,l,h,p,f,v,a)
Oracle : 
	nom (*$\ne$*) "Alex" (*$\land$*) nom (*$\ne$*) "Ryan"
	Une exception est levee
	
Cas de test : Personnage::testInitFailing 
CI : nom = "Alex" (*$\land$*) l = -5 (*$\land$*) h = 30 (*$\land$*) p = 30 (*$\land$*) a = 10 (*$\land$*) v = 100 (*$\land$*) f = 100
Operation : P0 (*$\stackrel{def}{=}$*) init(nom,l,h,p,f,v,a)
Oracle : 
	l (*$\leq$*) 0
	Une exception est levee
	
Cas de test : Personnage::testInitFailing 
CI : nom = "Alex" (*$\land$*) l = 30 (*$\land$*) h = -5 (*$\land$*) p = 30 (*$\land$*) a = 10 (*$\land$*) v = 100 (*$\land$*) f = 100
Operation : P0 (*$\stackrel{def}{=}$*) init(nom,l,h,p,f,v,a)
Oracle : 
	h (*$\leq$*) 0
	Une exception est levee
	
Cas de test : Personnage::testInitFailing 
CI : nom = "Alex" (*$\land$*) l = 30 (*$\land$*) h = 30 (*$\land$*) p = -5 (*$\land$*) a = 10 (*$\land$*) v = 100 (*$\land$*) f = 100
Operation : P0 (*$\stackrel{def}{=}$*) init(nom,l,h,p,f,v,a)
Oracle : 
	p (*$\leq$*) 0
	Une exception est levee
	
Cas de test : Personnage::testInitFailing 
CI : nom = "Alex" (*$\land$*) l = 30 (*$\land$*) h = 30 (*$\land$*) p = 30 (*$\land$*) a = -10 (*$\land$*) v = 100 (*$\land$*) f = 100
Operation : P0 (*$\stackrel{def}{=}$*) init(nom,l,h,p,f,v,a)
Oracle : 
	a < 0
	Une exception est levee
	
Cas de test : Personnage::testInitFailing 
CI : nom = "Alex" (*$\land$*) l = 30 (*$\land$*) h = 30 (*$\land$*) p = 30 (*$\land$*) a = 10 (*$\land$*) v = 0 (*$\land$*) f = 100
Operation : P0 (*$\stackrel{def}{=}$*) init(nom,l,h,p,f,v,a)
Oracle : 
	v (*$\leq$*) 0
	Une exception est levee
	
Cas de test : Personnage::testInitFailing 
CI : nom = "Alex" (*$\land$*) l = 30 (*$\land$*) h = 30 (*$\land$*) p
= 30 (*$\land$*) a = 10 (*$\land$*) v = 100 (*$\land$*) f = -8
Operation : P0 (*$\stackrel{def}{=}$*) init(nom,l,h,p,f,v,a)
Oracle : 
	f (*$\leq$*) 0
	Une exception est levee

Cas de test : Personnage::RetraitVieWorking
CI : personnage = init("Alex", 10, 10, 10, 10, 100, 100)
Operation : P0 (*$\stackrel{def}{=}$*) retraitPdV(personnage, 3)
Oracle :
	argent(P0) = 7 = (10-3)

Cas de test : Personnage::RetraitVieFailing
CI :personnage = init("Alex", 10, 10, 10, 10, 100, 100)
Operation : P0 (*$\stackrel{def}{=}$*) retraitPdV(personnage, -5);
Oracle : 
	-5 < 0
	Une exception est levee
	
	
	Cas de test : Personnage::RetraitArgentWorking
CI : personnage = init("Alex", 10, 10, 10, 10, 100, 100)
Operation : P0 (*$\stackrel{def}{=}$*) retraitArgent(personnage, 3)
Oracle :
	pointsDeVie(P0) = 7 = (10-3)

Cas de test : Personnage::RetraitArgentFailing
CI :personnage = init("Alex", 10, 10, 10, 10, 100, 100)
Operation : P0 (*$\stackrel{def}{=}$*) retraitArgent(personnage, -5);
Oracle : 
	-5 < 0
	Une exception est levee
	
Cas de test : Personnage::RetraitArgentFailing
CI :personnage = init("Alex", 10, 10, 10, 10, 100, 100)
Operation : P0 (*$\stackrel{def}{=}$*) retraitArgent(personnage, 108);
Oracle : 
	108 > 100
	Une exception est levee
	
Cas de test : Personnage::DepotAgentWorking
CI : personnage = init("Alex", 10, 10, 10, 10, 100, 100)
Operation : P0 (*$\stackrel{def}{=}$*) depotArgent(personnage, 3)
Oracle :
	pointsDeVie(P0) = 7 = (10-3)

Cas de test : Personnage::DepotArgentFailing
CI :personnage = init("Alex", 10, 10, 10, 10, 100, 100)
Operation : P0 (*$\stackrel{def}{=}$*) depotArgent(personnage, -5);
Oracle : 
	-5 < 0
	Une exception est levee
	
Cas de test : Personnage::ramasserObjetWorking
CI : personnage = init("Alex", 10, 10, 10, 10, 100, 100),
		 obj = Objet::init("arme",10,0)
Operation : P0 (*$\stackrel{def}{=}$*) ramasserObjet(personnage, obj)
Oracle :
	objetEquipe(personnage) = obj
	force(personnage) = 110

Cas de test : Personnage::ramasserObjetFailing1
CI :	personnage = init("Alex", 10, 10, 10, 10, 100, 100)
			obj = Objet::init("sous",0,40)
Operation : P0 (*$\stackrel{def}{=}$*) ramasserObjet(personnage, obj);
Oracle : 
	Objet:est_equipable(obj) = false
	Une exception est levee

Cas de test : Personnage::ramasserObjetFailing2
CI :	personnage = init("Alex", 10, 10, 10, 10, 100, 100)
			p = Personnage::init("Ryan",10,10,10,100,100)
			personnage = ramasser_perso(p)
			obj = Objet::init("arme",10,0)
Operation : P0 (*$\stackrel{def}{=}$*) ramasserObjet(personnage, obj);
Oracle : 
	est_equipe_perso(personnage) = true
	Une exception est levee
	
	
Cas de test : Personnage::ramasserArgentWorking
CI : personnage = init("Alex", 10, 10, 10, 10, 100, 100),
		 obj = Objet::init("piece",0,40)
Operation : P0 (*$\stackrel{def}{=}$*) ramasserObjet(personnage, obj)
Oracle :
	objetEquipe(personnage) = obj
	force(personnage) = 110

Cas de test : Personnage::ramasserArgentFailing1
CI :	personnage = init("Alex", 10, 10, 10, 10, 100, 100)
			personnage = retraitVie(personnage,200)
			obj = Objet::init("soussou",0,40)
Operation : P0 (*$\stackrel{def}{=}$*) ramasserArgent(personnage, obj)
Oracle : 
	estVaincu(P0) = true
	Une exception est levee
	
Cas de test : Personnage::ramasserArgentFailing2
CI :	personnage = init("Alex", 10, 10, 10, 10, 100, 100)
			obj = Objet::init("soussou",30,0)
Operation : P0 (*$\stackrel{def}{=}$*) ramasserArgent(personnage, obj);
Oracle : 
	Objet:est_DeValeur(obj) = false
	Une exception est levee
	
Cas de test : Personnage::ramasserPersoWorking
CI : personnage = init("Alex", 10, 10, 10, 10, 100, 100)
			perso2 = init("Ryan", 10, 10, 10, 10, 100, 100)
Operation : P0 (*$\stackrel{def}{=}$*) ramasserPerso(personnage, perso2)
Oracle :
	persoEquipe(P0) = perso2

Cas de test : Personnage::ramasserPersoFailing1
CI :	personnage = init("Alex", 10, 10, 10, 10, 100, 100)
			perso2 = init("Ryan", 10, 10, 10, 10, 100, 100)
			personnage = personnage.retraitPdV(10000);
Operation : P0 (*$\stackrel{def}{=}$*) ramasserPerso(personnage, perso2 )
Oracle : 
	estVaincu(P0) = true
	Une exception est levee
	
	
Cas de test : Personnage::jeterWorking
CI : personnage = init("Alex", 10, 10, 10, 10, 100, 100)
			obj = Objet::init("arme",10,0)
			personnage = ramasserObjet(personnage,obj)
Operation : P0 (*$\stackrel{def}{=}$*) jeter(personnage)
Oracle :
	persoEquipe(P0) = null
	force(P0) = 100
	objetEquipe(P0) = false

Cas de test : Personnage::jeterFailing1
CI :	personnage = init("Alex", 10, 10, 10, 10, 100, 100)
Operation : P0 (*$\stackrel{def}{=}$*) jeter(personnage)
Oracle : 
	estEquipeObjet(personnage) = false
	estEquipePerso(personnage) = false
	Une exception est levee
	
\end{Test}
\newpage
\section{Gangster}
\begin{Test}

Cas de test  : Gangster::testInitWorking
CI : nom = "Slick" (*$\land$*) l = 10 (*$\land$*) h = 10 (*$\land$*) p = 10 (*$\land$*) f = 10 (*$\land$*) v = 10
Operation : G0 (*$\stackrel{def}{=}$*) init(nom,l,h,p,f,v)
Oracle :
	nom(P0) = "Slick"
	largeur(P0) = 10
	hauteur(P0) = 10
	profondeur(P0) = 10
	pointsDeVie(P0) = 10
	force(P0) = 10
	argent(P0) = 0

Cas de test  : Gangster::testInitFailing1
CI : nom = "" (*$\land$*) l = 10 (*$\land$*) h = 10 (*$\land$*) p = 10 (*$\land$*) f = 10 (*$\land$*) v = 10
Operation : G0 (*$\stackrel{def}{=}$*) init(nom,l,h,p,f,v)
Oracle :
	nom(P0) == ""
	Une exception est levee
	
Cas de test  : Gangster::testInitFailing2
CI : nom = "Slick" (*$\land$*) l = -1 (*$\land$*) h = 10 (*$\land$*) p = 10 (*$\land$*) f = 10 (*$\land$*) v = 10
Operation : G0 (*$\stackrel{def}{=}$*) init(nom,l,h,p,f,v)
Oracle :
	l(*$\le$*)0
	Une exception est levee

Cas de test  : Gangster::testInitFailing3
CI : nom = "Slick" (*$\land$*) l = 10 (*$\land$*) h = -1 (*$\land$*) p = 10 (*$\land$*) f = 10 (*$\land$*) v = 10
Operation : G0 (*$\stackrel{def}{=}$*) init(nom,l,h,p,f,v)
Oracle :
	h (*$\le$*) 0
	Une exception est levee
	
Cas de test  : Gangster::testInitFailing4
CI : nom = "Slick" (*$\land$*) l = 10 (*$\land$*) h = 10 (*$\land$*) p = -1 (*$\land$*) f = 10 (*$\land$*) v = 10
Operation : G0 (*$\stackrel{def}{=}$*) init(nom,l,h,p,f,v)
Oracle :
	p (*$\le$*) 0
	Une exception est levee

Cas de test  : Gangster::testInitFailing5
CI : nom = "Slick" (*$\land$*) l = 10 (*$\land$*) h = 10 (*$\land$*) p = 10 (*$\land$*) f = -1 (*$\land$*) v = 10
Operation : G0 (*$\stackrel{def}{=}$*) init(nom,l,h,p,f,v)
Oracle :
	f (*$\le$*) 0
	Une exception est levee

Cas de test  : Gangster::testInitFailing6
CI : nom =  "Slick" (*$\land$*) l = 10 (*$\land$*) h = 10 (*$\land$*) p = 10 (*$\land$*) f = 10 (*$\land$*) v = -1
Operation : G0 (*$\stackrel{def}{=}$*) init(nom,l,h,p,f,v)
Oracle :
	v (*$\le$*) 0
	Une exception est levee


     
\end{Test}

\section{Bloc}
\begin{Test}

Cas de test  : Bloc::testInitWorking
CI : type = TYPE.VIDE (*$\land$*) o = null
Operation : B0 (*$\stackrel{def}{=}$*) init(type,o)
Oracle :
	type(B0) = TYPE.VIDE
	objet(B0) = 0

Cas de test  : Bloc::testInitFailing1
CI :  type = TYPE.VIDE (*$\land$*) o = Objet::init("machin",100,0)
Operation : B0 (*$\stackrel{def}{=}$*) init(type,o)
Oracle :
	type== TYPE.VIDE (*$\land$*) o (*$\ne$*) null
	Une exception est levee

Cas de test  : Bloc::testInitFailing2
CI :  type = TYPE.OBJET (*$\land$*) o = null
Operation : B0 (*$\stackrel{def}{=}$*) init(type,o)
Oracle :
	type== TYPE.OBJET (*$\land$*) o == null
	Une exception est levee

Cas de test : Bloc::RetirerObjetWorking
CI : b = init(TYPE.OBJET,Objet::init("machin,100,0))
Operation : B0 (*$\stackrel{def}{=}$*) retirerObjet(b)
Oracle :
	type(B0) = TYPE.VIDE
	objet(B0) = null

Cas de test : Bloc::RetirerObjetFailing
CI : b = init(TYPE.VIDE,null)
Operation : B0 (*$\stackrel{def}{=}$*) retirerObjet(b)
Oracle :
	type(b) (*$\ne$*) TYPE.OBJET
	Une exception est levee 

Cas de test : Bloc::PoserObjetWorking
CI : b = init(TYPE.VIDE,null) (*$\land$*) o = Objet::init("machin",100,0)
Operation : B0 (*$\stackrel{def}{=}$*) poserObjet(b,o)
Oracle :
	type(B0) = TYPE.OBJET
	objet(B0) = o

Cas de test : Bloc::PoserObjetFailing
CI : b = init(TYPE.OBJET,Objet::init("machin,100,0)) (*$\land$*) o = Objet::init("truc",10,0)
Operation : B0 (*$\stackrel{def}{=}$*) poserObjet(b,o)
Oracle :
	type(b) (*$\ne$*) TYPE.VIDE
	Une exception est levee 
\end{Test}
 
\section{Objet}
\begin{Test}

Cas de test  : Objet::testInitWorking1
CI : n = "machin" (*$\land$*) bonus = 10 (*$\land$*) valeur = 0
Operation : O0 (*$\stackrel{def}{=}$*) init(n,bonus,valeur)
Oracle :
	nom(O0) = "machin"
	bonus\_force(O0) = 10
	valeur\_marchande(O0) = 0
	
Cas de test  : Objet::testInitWorking2
CI : n = "machin" (*$\land$*) bonus = 0 (*$\land$*) valeur = 10
Operation : O0 (*$\stackrel{def}{=}$*) init(n,bonus,valeur)
Oracle :
	nom(O0) = "machin"
	bonus\_force(O0) = 0
	valeur\_marchande(O0) = 10
	
Cas de test  : Objet::testInitFailing1
CI : n = "" (*$\land$*) bonus = 0 (*$\land$*) valeur = 10
Operation : O0 (*$\stackrel{def}{=}$*) init(n,bonus,valeur)
Oracle :
	n == ""
	Une exception est levee
	
Cas de test  : Objet::testInitFailing2
CI : n = "truc" (*$\land$*) bonus = 10 (*$\land$*) valeur = 10
Operation : O0 (*$\stackrel{def}{=}$*) init(n,bonus,valeur)
Oracle :
	bonus > 0 (*$\land$*) valeur > 0
	Une exception est levee
	
Cas de test  : Objet::testInitFailing3
CI : n = "truc" (*$\land$*) bonus = -1 (*$\land$*) valeur = 0
Operation : O0 (*$\stackrel{def}{=}$*) init(n,bonus,valeur)
Oracle :
	bonus < 0
	Une exception est levee
	
Cas de test  : Objet::testInitFailing4
CI : n = "truc" (*$\land$*) bonus = 0 (*$\land$*) valeur = -1
Operation : O0 (*$\stackrel{def}{=}$*) init(n,bonus,valeur)
Oracle :
	valeur < 0
	Une exception est levee


\end{Test}

\section{Terrain}
\begin{Test}
Cas de test  : Terrain::testInitWorking
CI : p = 50, h = 100, l = 50
Operation : T0 (*$\stackrel{def}{=}$*) init(l,h,p)
Oracle :
	largeur(T0) = l
	hauteur(T0) = h
	profondeur(T0) = p

Cas de test  : Terrain::testInitFailing1
CI :  p = 0, h = 100, l = 50
Operation : T0 (*$\stackrel{def}{=}$*) init(l,h,p)
Oracle :
	p<50
	Une exception est levee

Cas de test  : Terrain::testInitFailing2
CI :  p = 50, h = 100, l = 0
Operation : T0 (*$\stackrel{def}{=}$*) init(l,h,p)
Oracle :
	l<50
	Une exception est levee

Cas de test  : Terrain::testInitFailing3
CI :  p = 50, h = 10, l = 50
Operation : T0 (*$\stackrel{def}{=}$*) init(l,h,p)
Oracle :
	h<100
	Une exception est levee

Cas de test  : Terrain::testInitFailing4
CI :  p = 60, h = 100, l = 50
Operation : T0 (*$\stackrel{def}{=}$*) init(l,h,p)
Oracle :
	(p%50)(*$\ne$*)0
	Une exception est levee

Cas de test  : Terrain::testInitFailing5
CI :  p = 50, h = 100, l = 60
Operation : T0 (*$\stackrel{def}{=}$*) init(l,h,p)
Oracle :
	(l%50)(*$\ne$*)0
	Une exception est levee

Cas de test : Terrain::modifierBlocWorking
CI : t = init(500,100,500) (*$\land$*) b = Bloc::init(TYPE.VIDE,null)
Operation T0 (*$\stackrel{def}{=}$*) modifierBloc(t,10,10, b)
Oracle :
	bloc(T0,10,10) = b

Cas de test : Terrain::modifierBlocFailing1
CI : t = init(500,100,500) (*$\land$*) b = Bloc::init(TYPE.VIDE,null)
Operation T0 (*$\stackrel{def}{=}$*) modifierBloc(t,-1,10, b)
Oracle :
	x < 0
	Une exception est levee

Cas de test : Terrain::modifierBlocFailing2
CI : t = init(500,100,500) (*$\land$*) b = Bloc::init(TYPE.VIDE,null)
Operation T0 (*$\stackrel{def}{=}$*) modifierBloc(t,1000,10, b)
Oracle :
	x > largeur(t)
	Une exception est levee

Cas de test : Terrain::modifierBlocFailing3
CI : t = init(500,100,500) (*$\land$*) b = Bloc::init(TYPE.VIDE,null)
Operation T0 (*$\stackrel{def}{=}$*) modifierBloc(t, 10,-1, b)
Oracle :
	y < 0
	Une exception est levee 
 
Cas de test : Terrain::modifierBlocFailing4
CI : t = init(500,100,500) (*$\land$*) b = Bloc::init(TYPE.VIDE,null)
Operation T0 (*$\stackrel{def}{=}$*) modifierBloc(t, 10,1000, b)
Oracle :
	y > profondeur(t)
	Une exception est levee 

Cas de test : Terrain::modifierBlocFailing5
CI : t = init(500,100,500) (*$\land$*) b =null
Operation T0 (*$\stackrel{def}{=}$*) modifierBloc(t, 10,10, b)
Oracle :
	b == null
	Une exception est levee 

	    
\end{Test}


\section{Moteur de jeu}
\begin{Test}
	    
\end{Test}
\section{GestionCombat}
\begin{Test}
	

\end{Test}

\end{document}
