\documentclass[12pt,francais]{beamer}

\institute{UPMC}
 
\useinnertheme{rounded} % blocks arondis
\usecolortheme{rose} %permet les cadre autour des blocks

\setbeamertemplate{blocks}[rounded][shadow=true] % ombre block

\useoutertheme{smoothbars}

%\setbeamercolor{separation line}{use=structure,bg=structure.fg!10!bg} % ligne de séparation

%\useoutertheme[footline=institutetitle]{miniframes} % ligne de sous section + ligne du bas

\useoutertheme{miniframes} % ligne de sous section

\setbeamertemplate{frametitle}[default][center] % centrer le titre

\DeclareOptionBeamer{subsection}[false]{\csname beamer@theme@subsection#1\endcsname}

\setbeamercolor{section in head/foot}{use=structure,bg=structure.fg!25!bg} % permet dégradé
\AtBeginDocument{%permet le dégradé en haut
  {
    \usebeamercolor{section in head/foot}
  }
  
  \pgfdeclareverticalshading{beamer@headfade}{\paperwidth}
  {%
    color(0cm)=(bg);
    color(1.25cm)=(section in head/foot.bg)%
  }

  \setbeamercolor{section in head/foot}{bg=}
}
\addtoheadtemplate{\pgfuseshading{beamer@headfade}\vskip-1.25cm}{} % dégradé

\beamertemplatedotitem

\DeclareOptionBeamer{compress}{\beamer@compresstrue} % ??
\ProcessOptionsBeamer % ??

\setbeamertemplate{navigation symbols}{} % suppression des symbols de nav

% ajout de la barre du bas
\setbeamertemplate{footline}{
\leavevmode%
\hbox{\hspace*{-0.06cm}
\begin{beamercolorbox}[wd=.33\paperwidth,ht=2.25ex,dp=1ex,left]{author in head/foot}%
        \usebeamerfont{author in head/foot}\insertshortauthor~~(\insertshortinstitute)
\end{beamercolorbox}%
\begin{beamercolorbox}[wd=.33\paperwidth,ht=2.25ex,dp=1ex,center]{section in head/foot}%
        \usebeamerfont{section in head/foot}\insertshorttitle
\end{beamercolorbox}%
\begin{beamercolorbox}[wd=.33\paperwidth,ht=2.25ex,dp=1ex,right]{section in head/foot}%
        \usebeamerfont{section in head/foot}\insertshortdate{}\hspace*{2em}
        \insertframenumber{} / \inserttotalframenumber\hspace*{2ex}
\end{beamercolorbox}}%
\vskip0pt%
}

%theme "ADELE"
\definecolor{ADEL_BLEU_FONCE}{RGB}{72,148,163}
\definecolor{ADEL_GRIS}{RGB}{171,185,187}
\definecolor{ADEL_BLEU_CLAIRE}{RGB}{193,218,227}

\colorlet{beamer@blendedblue}{ADEL_BLEU_FONCE!60!black}

%\beamerdefaultoverlayspecification{<+->}

\usepackage{multicol} %columnes
\usepackage{colortbl}


\usepackage{graphicx}

\usepackage{amsmath}
\usepackage{amsfonts}
\usepackage{amssymb}


\usepackage{listings}                                                
\lstset{language={Clean}}

\usepackage[utf8]{inputenc}
\usepackage[T1]{fontenc}
\usepackage{lmodern}
\usepackage[francais]{babel}

%%%%%%%%%%% Sommaire beamer %%%%%%%%%%%
\AtBeginSection[]{
  \begin{frame}{Plan}
    \small \tableofcontents[currentsection, hideothersubsections]
        \end{frame}
}


\title{Huffman Adaptatif}
\author{Badoual Yannick, Le Frioux Ludovic}


\begin{document}

\begin{frame}
        \frametitle{\includegraphics[width=3.4cm]{logo.png}}
        \begin{center}
                Huffman Adaptatif,\\
                \vspace{1cm}
                vous est pr\'esent\'e par Badoual Yannick et Le Frioux Ludovic
        \end{center}
\end{frame}

\begin{frame}
\frametitle{\includegraphics[width=3.4cm]{logo.png}\\Plan}
  \tableofcontents
\end{frame}


\section{Pr\'esentation}
\begin{frame}
  \frametitle{Pr\'esentation}
  \begin{block}{Huffman Adaptatif}
    \begin{itemize}
      \item Algorithme de compression
      \item Adaptatif / Dynamique : streams
    \end{itemize}
  \end{block}
  \begin{block}{Principe}
    \begin{itemize}
      \item Arbre binaire entier : 0 ou 2 fils
      \item 1 char => 1 feuille
      \item Position dans l'arbre => codage
      \item + fréquent => + haut dans l'arbre => + court code
    \end{itemize}
  \end{block}
\end{frame}

\section{Complexité}
\begin{frame}
  \frametitle{Complexité}
        \vspace{-1cm}
        \center $n = |\Sigma|$
  \begin{block}{Temps}
    \begin{itemize}
      \item Pour chaque insertion : $O(n)$
      \item Texte longueur $t, O(t \times n)$
    \end{itemize}
  \end{block}
  \begin{block}{Espace}
    \begin{itemize}
      \item 1 char => 1 feuille
      \item $2n + 1 n\oe uds$
      \item $O(n)$
    \end{itemize}
  \end{block}
\end{frame}

\section{Implantation}
\begin{frame}
  \frametitle{Implantation}
        \vspace{-1cm}
        \center Java
  \begin{block}{Conception}
    \begin{itemize}
      \item Arbre binaire : Node, InternalNode, Leaf
      \item HuffmanTree : Modélise l'arbre, traitement, modification
    \end{itemize}
  \end{block}
  \begin{block}{IO}
    \begin{itemize}
      \item Utilisation InputStream, OutputStream
      \item Possibilité d'utiliser des socket, ou autre streams
      \item Restriction dû à l'état actuel de l'arbre (nombre de bits requis)
    \end{itemize}
  \end{block}
\end{frame}

\section{Analyse}
\begin{frame}
  \frametitle{Analyse}
  \begin{block}{Taux de compression}
    \begin{itemize}
      \item Sur des textes $\approx 50\%$
      \item Dans un texte normal $\approx 40e de caractères \approx 4 bits$
       \item Sur autres fichiers (mp3, mkv) $< 1\%$
    \end{itemize}
  \end{block}
  \begin{block}{Temps}
    \begin{itemize}
      \item Théorique $O(t\times n)$
      \item Pratique : $O(t \times log(n))$
      \item Car + fréquent => code + court
    \end{itemize}
  \end{block}
\end{frame}

\section{Conclusion}
\begin{frame}
  \frametitle{Conclusion}
  \begin{block}{Conclusion}
    \begin{itemize}
      \item Marche très bien sur des textes
      \item Sans intérêts sur musiques, films
       \item Peut fonctionner sur des streams
       \item Sur des fichiers textes : Meilleur compression en semi-adaptatif
    \end{itemize}
  \end{block}
\end{frame}

\end{document}

